
\chapter{Application of the POD-DEIM approach to a marine ecosystem model} % Main chapter title

\label{Chapter3} % For referencing the chapter elsewhere, use \ref{Chapter1} 

%----------------------------------------------------------------------------------------

\section{Introduction}
To apply the POD-DEIM approach on a marine ecosystem model, the Simulation Package Metos3D (a Marine Ecosystem Toolkit for Optimization and
Simulation in 3-D) \cite{metos3dsimpack} is used. This package provides the possibility to simulate the ocean circulation coupled with biogeochemical processes. 
For the simulation of the ocean circulation the package uses transport matrices as described in \cite{Khatiwala2005}. 
The matrices are extracted from a general circulation model (GCM) and are capturing the complex three-dimensional
advective diffusive transport of tracers in a GCM as a sparse matrix. 
The so called transport matrix method (TMM), demonstrated in \cite{TMM}, makes use of these pre-computed transport data to apply it on arbitrary variables. 
Monthly averaged diffusion and advection matrices, which are interpolated during computation, are used as a compromise between storage and accuracy.
The effects, that are in longitudinal and latitudinal directions, are treated explicitly while the vertical effects are computed implicitly. 
Thus, there are twelve explicit and twelve implicit matrices.  

In addition to this linear part, the package provides an interface for different 
biogeochemical models that vary in the number of tracers and types of modeled processes.
They are controlled by a set of parameters $u$ and describe processes like nutrient consumption as well as growing, dying and sinking of different tracer elements.
For example, the N-Model, that is used in Chapter \ref{Chapter4}, calculates the uptake of inorganic nutrients by phytoplankton of an implicit pyhtoplankton concentration in the euphotic zone (cf. \cite{biomodels}). 
This uptake is limited by the amount available inorganic 
nutrients and by the light. The model uses phosphate ($PO_4$) as tracer element which governs the rate of growth of pyhtoplankton.  


The surface grid has a resolution of $2.8125 ^{\circ}$ in both directions, longitudinal and latitudinal. This results into a $128 \times 64$ 
grid with 4448 horizontal points in the oceans, if the grid points, where only land is, are left out. Each grid point represents a water column, called profile. Each profile has at most 15 vertical depth layers.
Overall, there are $ n_h = 52749 $ grid points, which are held in 
one single vector for each tracer. The implicit and explicit matrices are extremely sparse as they have only 672k (0,024\%) and 5407k (0,19\%) non zero values. To solve a system with more than one tracer, the matrices and tracers 
are arranged in a block diagonal system. Thus, with each tracer the linear system grows.

The temporal resolution is $\Delta t = 1/2880$. Assuming a year with 360 days, $\Delta t$ represent a step of three hours. The boundary data is used to have the ice cover and latitude information for each profile and a domain data set
provides the model with the information of the depths and heights of the vertical layers of each profile. 

The packed provides two solver for periodic steady-states one is a fixed point iteration (spin-up) and the other a Newton solver. In this thesis only the spin-up solver is used.


\section{Reduce the System}
As said, the full discrete representation of the iteration for n tracers is a block diagonal system. An iteration
reads
\begin{equation}\label{metos_equ}
y_{j+1} = A_{impj} (A_{expj} y_j + \Delta t q_j(y_j,u,b_j,d_j)),
\end{equation}
where $y_j = (y_i(t_j))^n_{i=1}$ is a vector combined of all n tracer vectors and $t_j = t_0 + (j-1)* \Delta t$ is a point in time.
$A_{impj}$ and $ A_{expj}$ are the
block diagonal matrices, interpolated for the time index $j$. The discrete boundary and domain data is represented by
$b_j = (b_i(t_j))^{n_b}_{i=1}$ and $d_j = (d_i(t_j))^{n_d}_{i=1}$. The biogeochemical model $q_j$ is described by
$q_j(y_j,u,b_j,d_j) = (q_i(t_j,y_i,u,b_i,d_i))^n_{i=1}$.

To apply the POD-DEIM approach, let $n=1$, so that there is only one tracer and let $n_h$ be the length of the tracer vector. The basis matrices $V \in \mathbb{R}^{n_h \times k}$
and $U \in \mathbb{R}^{n_h \times m}$ are computed with the POD method. The projector $\mathbb{P} = U(P^TU)^{-1}P^T$ is computed
with the result $P \in \mathbb{R}^{n_h \times m}$ of the DEIM-Algorithm with input $U$.

After projecting the system \eqref{metos_equ} onto $V$ and applying the DEIM approximation onto the nonlinear part, it
reads
\begin{equation}\label{metos_equ_projected}
 \hat{y}_{j+1} = V^TA_{impj} (A_{expj}V \hat{y}_j + \Delta t    \mathbb{P} q_j(V \hat{y}_j,u,b_j,d_j)),
\end{equation}
where $\hat{y}_j = V^T y_j$.

To use the advantage of the reduction, $V^TA_{impj}$ is multiplied out and the system looks like this:
\begin{equation}\label{metos_red}
\begin{aligned}
 \hat{y}_{j+1} &= \underbrace{V^TA_{impj} A_{expj}V}_{A_{rj} \in \mathbb{R}^{k \times k}} 
 \hat{y}_j + \Delta t  \underbrace{V^TA_{impj} U(P^TU)^{-1}}_{\mathbb{P}_rj \in \mathbb{R}^{k \times m}} P^T q_j(V \hat{y}_j,u,b_j,d_j)
  \\ &= A_{rj} \hat{y}_j + \Delta t    \mathbb{P}_{rj} P^T q_j(V \hat{y}_j,u,b_j,d_j).
 \end{aligned}
\end{equation}
 
The matrices $A_{rj}$ and $\mathbb{P}_{rj}$ are interpolated out of the twelve monthly average matrices $A_{ri}$ and $\mathbb{P}_{ri}$ for $i = 1,...,12$.
$A_{ri}$ and $\mathbb{P}_{ri}$ are precomputed out of monthly averaged matrices $A_{impi}$ and $A_{expi}$ of the original system,
the POD-basis $V$ as well as $U$ and the DEIM index matrix $P$. Thus, $A_{ri} = V^T A_{impi} A_{expi} V$ and
$\mathbb{P}_{ri} = V^T A_{impi} U(P^TU)^{-1}$. They are dense, but quite small, because $k<<n_h$ and $m<<n_h$.

\subsection{Reduction of the nonlinear part}
The reduction of the nonlinear part is more complex. The best reduction would be possible if the nonlinear function could be evaluated componentwise. Then, only 
the selected points have to be computed and the reduction would be from $n_h$ to $m$ evaluations.
But in Metos3D the biogeochemical model is evaluated independently for each profile.


Thus, to compute one point a whole profile has to be evaluated. The profiles that have to be evaluated can be calculated with 
the help of the DEIM-indices and the land-sea mask which contains the geometry information.
From all the $p=4448$ profiles of the full-order model (FOM), only $m$ are computed. Thus, the dimension reduction is less
than the reduction of the linear part but one evaluation of the biogeochemical model can be very expensive.  
 
The following pseudo code describes the reduced evaluation of the nonlinear function.
\begin{algorithm}
\caption{Pseudo code for the reduction of a biogeochemical model}\label{Chapter3:deim_algorithm}
\begin{algorithmic}[1]
   \item \textbf{Input: }{$\vec{\varrho} = \{\varrho_1,..,\varrho_m\}^T \in \mathbb{N}^m$}
   \item \textbf{Output: }{$\hat{q} \in \mathbb{R}^m$} 
   \For{$i=0 $ to $m$}
      \State $index = \textbf{range of profile in tracer vector for DEIM-point } \varrho[i]$ 
      \State $y_p = V[index,:] \hat{y}$ 
      \State $b_p,d_p$ = \textbf{select corresponding boundary and domain conditions}
      \State $q_p = q(y_p,u,b_p,d_p)$
      \State $\hat{q}[i] = q_p[\varrho [i]]$ 
    \EndFor
\end{algorithmic}
\end{algorithm}

This algorithm uses the DEIM-indices as input and provides an evaluation of the biogeochemical model at the desired points.
In line 4 the range of the profile for a specific DEIM-point is determined, and then, this part of the high dimensional tracer
is computed, through projection with the POD-basis $V$. After selecting the corresponding boundary and domain conditions
the biogeochemical model function is evaluated for this profile. In line 8, the desired point in the profile is
saved and the rest of the vector is discarded. 


Overall, for one time step of the reduced model, the steps that are described by Algorithm \ref{Chapter3:reduced_algorithm} have to be done.
\begin{algorithm}
\caption{pseudo code for time step of reduced model}\label{Chapter3:reduced_algorithm}
\begin{algorithmic}[1]
  \item \textbf{Input: } {$\textbf{point in time } t \in [0,1)$, $\hat{y}_j \in \mathbb{R}^k$}
 \item \textbf{Output: } {$\hat{y}_{j+1} \in \mathbb{R}^k $}
 \State \textbf{compute $\hat{q}$ with Algorithm \ref{Chapter3:deim_algorithm}}
 \State $\textbf{ interpolate } A_{ri} \textbf{ and } \mathbb{P}_{ri} \textbf{ in point } t$
 \State $\hat{y}_{j+1} = A_{ri}\hat{y}_j + \mathbb{P}_{ri} \hat{q}$ 
\end{algorithmic}
\end{algorithm}

For one time step, the desired parts of the biogeochemical model have to be evaluated, and then, two matrix
vector multiplications have to be done. 


\section{Implementation}
The code to generate the reduced-order models was mainly written in python using scipy and numpy \cite{scipy}. The DEIM Algorithm and the precomputing of the system matrices were implemented
with numpy. To exchange data with the PETSc library, used by Metos3D, IO routines for PETSc files were created in python. 

To compute the SVD of the snapshot matrices, at first, numpy was used. 
As the matrices have become quite large during the progress of this thesis, and the SVD algorithm in numpy was not suitable
for matrices of this size, the library SLEPc \cite{slepc-toms} was used.
This provides the possibility to compute partial SVDs in parallel. The algorithms SLEPc uses are explained in \cite{svd_parallel}, where a performance analysis is given as well. 

The reduced model was implemented in python. The biogeochemical models used in Metos3D were included directly with F2PY (Fortran to Python interface generator).

The whole code is freely accessibly via
\begin{equation*}
 \text{\url{https://github.com/neeljp/MasterThesis_POD-DEIM}}
\end{equation*}
and can be used for further improvements or demonstration.











 












