
\chapter{Conclusion and Future Work} % Main chapter title

\label{Chapter6} % For referencing the chapter elsewhere, use \ref{Chapter1} 

%----------------------------------------------------------------------------------------

\section{Conclusion}
In this thesis, the nonlinear MOR method POD-DEIM has been applied to three different marine ecosystem models of the software package Metos3D.
Therefore, the reduced model equations \eqref{metos_red} were established and implemented. Then, an analysis of the one
tracer model, the N-Model, was made, in which the feasibility was tested. 
For that the manner of how many information from the full-order model are needed for an accurate reduced-order model was investigated.
Moreover, it was examined how the size of the reduced bases affect the accuracy and CPU time of the reduced-order model.
On the basis of these results six reduced-order models of the N-Model were compared to the full-order model.
The approach was extended on the two tracer model, the N-DOP-Model, and two reduced-order models were compared to the full-order N-DOP-Model.
Afterwards, it was tried to create reduced-order models of the N-Model, which could be used for parameter studies.
For that two approaches were tested. The first one was to create a reduced-order model out of multiple trajectories created with different parameter sets.
And the other one was to use one reduced-order model for multiple parameter sets.


One main goal of this thesis was to demonstrate the feasibility of the nonlinear MOR method POD-DEIM on 
marine ecosystem models with respect to accuracy and computational speed up. The numerical experiments in Chapter \ref{Chapter4}
have demonstrated the feasibility for a fixed set of parameters. The POD-DEIM approach leads to a significant speed up and a solution that is close to that of the full-order model measured in the 2-norm. For the two tracer N-DOP-Model the speed up is smaller because of
the fact that a base for each tracer has to be used to get an accurate reduced-order model.


The idea of using the POD-DEIM approach to get a reduced-order model for parameter studies has turned out to be not practical.
The main reason for that is diversity of the solutions for different sets of parameters, and thus, 
the approximation within a low dimensional subspace is very inaccurate.


\section{Future Work}
At this point, some directions for future research, that might build
up on the work of this thesis, are presented. 

\subsubsection*{Selection of snapshots}


In Chapter \ref{Chapter4_N} the selection of snapshots for the SVD is not optimal and could be further improved.
Since there is no theoretical approach to optimize the selection, it is only possible with numerical experiments.
A problem in that area is the soft upper bound of the total number of snapshots that is given by the SVD algorithm.
For that the Algorithm 5.1 in \cite{PODDeimapplication} may be interesting.
This algorithm uses the Gram-Schmidt QR factorization with a reorthogonalization algorithm from \cite{Daniel2007ReorthogonalizationAS}.
The algorithm detects and eliminates redundant snapshots as well as prevents the loss of orthogonality of the POD basis due to the numerical instability of the Gram–Schmidt process.


\subsubsection*{Creating a reduced-order model for parameter studies}
For parameter studies it could be tried to create a reduced-order model only for a subset of the parameters, thus, changing only some parameters and leave the others fixed.
This could lead to a usable model since the space spanned by the solution should be smaller. Another idea is to create a reduced-order model for a subspace of the parameter space, thus, use all parameters but only create a model for 
parameters in the range of $u_r \pm x\%$. The idea is the same, the reduction of the solution space, and thus, a better low dimensional approximation. 

\subsubsection*{Create combined ROMs}
After the experiments that are showed in Appendix \ref{A:last_year}, two ideas came up. 
Since the convergence of the ROM presented in Appendix \ref{A:last_year} is much faster than the convergence 
of the ROMs presented in Chapter \ref{behaviorofroms}, the first idea is to increase the overall speed up by using 
multiple of these ROMs. Each ROM has to be created with a  full trajectory of a different  model year, i.e. 500, 1000, 2000, 3000. 
Then, the first model is initialized with a starting concentration and simulated until it has reached the point, where
it starts to diverge. Afterwards, the next ROM is initialized with the solution of the previous one and simulated until 
they have reached their point of divergence. This could lead to an increase of the overall speed up and could increase the accuracy
as well, if enough ROMs are used. 

The other idea is to use a ROM, as in Appendix \ref{A:last_year}, 
in the opposite way it is used in Chapter \ref{behaviorofroms}. There, the ROMs were executed first. Thereafter 
the solution it has generated was improved with the full-order model. It could be feasible to run the FOM until 1000 or 2000 model years, and then, supply this 
solution into a ROM as in Appendix \ref{A:last_year}. Being initialized with a solution from that point the ROM converges much faster than the FOM.
After executing the ROM the solution could also be improved with the FOM.

The benefit of these two approaches is that a lot of spin-ups are ``skipped'', and thus, the overall speed up would be much higher than it is presented in Chapter \ref{behaviorofroms}.



