% Chapter 1

\chapter{Introduction} % Main chapter title

\label{Chapter1} % For referencing the chapter elsewhere, use \ref{Chapter1} 

%----------------------------------------------------------------------------------------

% Define some commands to keep the formatting separated from the content 
\newcommand{\keyword}[1]{\textbf{#1}}
\newcommand{\tabhead}[1]{\textbf{#1}}
\newcommand{\code}[1]{\texttt{#1}}
\newcommand{\file}[1]{\texttt{\bfseries#1}}
\newcommand{\option}[1]{\texttt{\itshape#1}}

%----------------------------------------------------------------------------------------

\section{Motivation}
Numerical simulations of dynamical systems are extremely successful for the means of studying complex physical, biological or chemical
phenomena. However the spatial and temporal dimensionality of those simulations often increase to the limits of computational requirements, storage and time.
One possible approach to overcome this is model reduction.

Model reduction has become imported for many fields of physics as they offer the potential to simulate dynamical systems with
a substantially increased computation efficiency. The technique is a mathematical approach to find a low-dimensional 
approximation for a system of ordinary differential equations. The goal is to determine a low-dimensional system that has the same
solution characteristics as the original system, but needs less storage and evaluation time.

This approach is used in particular for models, that model the physics of the complex earths climate 
and have large spatial and temporal scales. Thus if they provide a certain accuracy, they need a lot of computation time. 
For these systems it is a balancing act between computation time and accuracy in time and space, and
usually the computational resources are the limiting factor. 

In addition these models are often controlled by a
set of parameters and a common problem is to find a certain set of parameters. Which results in an, in some way, optimal solution,
for example in a solution that is close to real measurements. This search for a set of parameters requires, in most cases, many evaluations of the model, so
the computational cost is potentially higher. The idea is to use a reduced model for parts of this parameter search, to
lower the computational time.

In this thesis, model reduction of a marine ecosystem model will be investigated. Marine ecosystem models play an imported role 
in research of the global carbon cycle as well as for questions on climate change. They have huge spatial scale if they provide 
a certain accuracy. The resulting  partial differential equations are in there discrete form high in dimension. Also the temporal 
domain is large, because effects in the ocean take several thousand years to fully spread and thus these models need this time to approach
a steady-state. This results from 
the circulation circle of the ocean. It is estimated that it can take up to 1000 years for a parcel of water to complete one circle.
In addition to the ocean circulation a marine biogeochemistry model describes the marine chemistry, biology and the 
geological processes that occur in the ocean. These processes are controlled by a set of parameter of which some are not directly measurable.
Thus, parameter optimization is needed to fit the model output to real observations. Therefore, there is a strong demand
for model reduction for this kind of model. 


\section{Existing Methods}
Model Reduction or Model Order Reduction (MOR) was developed in the context of linear control theory. It is today
a wide field research in system and control theory and numerical mathematics such as signal analysis and
fluid dynamics. Most work has been done for linear MOR and the generally used methods are discussed in \cite{MOR2008}.
Among them the so-called Proper Orthogonal Decomposition (POD) also known as Karhunen-Loeve decomposition or Principal Component Analysis.
POD has been successfully applied to a wide field of research including pattern recognition \cite{PCAimagepattern,patternrec} , fluid dynamics \cite{A_anoptimizing,PODSWF}, turbulent flows \cite{Wang201210,Berkooz93theproper} and other numerical simulations \cite{Luo2016433,Buchan2015138}.
The POD method combined with Galerkin projection is a popular approach for constructing reduced-order models of PDEs. This approach has provided reduced-order models in many applications such as fluid dynamics, aerodynamics and optimal control.
Also it has been used for large scale systems with a high dimensional parametric input space. According to \cite{PODover},
where also a historical overview of the POD approach can be found, the first usage of POD for MOR of a 
dynamical system was in the 1990s, cf. \cite{PODfirst}.

The POD-Galerkin method only reduces linear parts and therefore another method is needed to reduce the complexity of the nonlinear terms.
Discrete Empirical Interpolation Method (DEIM) proposed by \cite{Chaturantabut2010Deim,PHDCha} expands the POD approach to construct
an interpolation of the nonlinear part. It is a discrete simplification of the Empirical Interpolation Method
(EIM) introduced by \cite{EIM}. The POD-DEIM approach has been applied to dynamical processes \cite{PODDeimapplication,PODDeimshallowwater,PODdeimpredetor,Xiao20141} and has
shown that it leads to a computational gain in complexity. 

The marine ecosystem model, which will be used here, has been introduced by \cite{metosidea}. The simulation
package Metos3D, that implements the model, is described briefly in Chapter \ref{Chapter3} and a full description is presented in \cite{metos3dsimpack}. 

\section{Outline} 
The work in this thesis is structured as follows: 
In Chapter 2 the problem is formulated and the POD approach and its extension the DEIM is explained. The next chapter 
introduces into the model behind Metos3D and shows the theoretically application of the POD-DEIM approach on this model.
Chapter 4 presents the results of the numerical experiments with some of the models Metos3D provides. It starts with an 
extensive analysis of the simplest model, called N-Model. Afterwards, a more complex model with two tracers is presented, 
the N-DOP-model. The results of the experiments for parameters studies with reduced-order models of the N-Model are content of 
Chapter 5.
The last chapter presents a conclusion and an outlook on future research questions that had come up
during this thesis, but were not considered.







