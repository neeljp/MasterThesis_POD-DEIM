% Chapter 2

\chapter{Nonlinear Model Reduction with POD-DEIM} % Main chapter title

\label{Chapter2} % For referencing the chapter elsewhere, use \ref{Chapter1} 

This chapter presents a model reduction method for nonlinear ordinary differential equation
(ODEs). First a general formulation of the problem is given and its reduction via Proper Orthogonal Decomposition
(POD) with Galerkin projection is shown. Continued with a discussion of the complexity issue in the nonlinear part.
To resolve this issue the Discrete Empirical Interpolation Method (DEIM) is introduced in \ref{Chapter2:deim}.
This method extends the orthogonal projection of POD in the nonlinear part with an interpolation of the nonlinear function.
The DEIM was fist introduced by Saifon Chaturantabut in \cite{Chaturantabut2010Deim}, where also the algorithm
and the error bound analysis presented in \ref{Chapter2:deim} comes from.

\section{Problem Formulation}
\label{Chapter2:problem}
Consider a system of nonlinear differential equations of the form
\begin{equation}
\label{problem:ODE}
 \frac{d}{dt}y(t) = Ay(t) + q(y(t))
\end{equation}
where $t \in [0,T]$ denotes the time, $y(t) = [y_1(t),\dots,y_n(t)]^T \in \mathbb{R}^n$ is a vector of states with
initial condition $y(0) = y_0 \in \mathbb{R}^n$ , $A \in \mathbb{R}^{n \times n}$ is the discrete approximation of the 
linear differential operator and $q:\mathbb{R}^n \longmapsto \mathbb{R}^n$ a nonlinear vector-based function.
The complexity, both time and space, to solve the system is mostly depended on the dimension $n$ of the system. Which could be very large, if
high accuracy is required, thus, to reduce the complexity, we want to reduce the dimension.

In order to do  this the system will be projected on a subspace spanned in $\mathbb{R}^n$ by a reduced basis of a dimension $k \ll n$.
These projection-based techniques are commonly used for constructing a reduced-order system, that approximates the
original system in a subspace. Here a Galerkin projection is used as the means for dimension reduction.
For that let $V \in \mathbb{R}^{n \times k}$ be a matrix whose columns are the orthonormal vectors of the reduced
basis. Then projecting the system \eqref{problem:ODE} onto $V$ by replacing $y(t)$ with $V\hat{y}(t)$, where $\hat{y}(t)$ 
$\in \mathbb{R}^k$, is the reduced system of the form
\begin{equation}\label{Chapter2:problem_pod}
  \frac{d}{dt}\hat{y}(t) = V^TAV\hat{y}(t) + V^Tq(V\hat{y}(t)).
\end{equation}
The quality of the approximation is clearly affected by the choice of the reduced basis vectors in $V$. The POD method constructs a set of basis
vectors from a singular value decomposition (SVD) of a set of snapshots
\begin{equation}
 S = \{y_1,\dots,y_m\},
\end{equation}
which contains samples of trajectories $y(\cdot)$ (called snapshots) for a particular set of parameters, boundary condition and other system inputs. 
After a reduced model has
been constructed from this basis, it may be used to compute approximate solutions for different initial conditions
and parameter settings. If the snapshots are diverse enough it is expected that the approximate solution is near to the high dimensional one.
The POD basis is optimal in the sense that a approximation error in relation to the snapshots is minimized. 
Therefore, the POD approach is used here for constructing the basis.

In the equation \eqref{Chapter2:problem_pod} the linear part is already reduced in dimension, 
because the pre-computation of $\hat{A}= V^TAV \in \mathbb{R}^{k \times k}$ supplies a system of dimension $k \ll n$.
But the evaluation of the nonlinear function $q(y(t))$ is still in the dimension of $n$, i. e.
\begin{equation}\label{Chapter2:problem_pod_dim}
  \frac{d}{dt}\hat{y}(t) = \underbrace{\hat{A}}_{k \times k}\hat{y}(t)+ V^Tq(\underbrace{V\hat{y}(t)}_{n}).
\end{equation}
To reduce the nonlinear part as well, the DEIM will be used. It provides an approximation $\hat{q}(y(t))$
of the nonlinear function by approximating the function $q(y(t))$ in an low dimensional subspace spanned by another basis $U \in \mathbb{R}^{n \times m}$, which is 
constructed out of snapshots of the nonlinear function with the 
POD method.
With coefficients $c(y(t))$ the approximation is of the form
\begin{equation}
 q(y(t)) \approx \hat{q}(y(t)) = Uc(y(t))
\end{equation}
and the form of the reduced system is
\begin{equation}
  \frac{d}{dt}\hat{y}(t) = \underbrace{\hat{A}}_{k \times k}\hat{y}(t)+ \underbrace{V^TU}_{k \times m} c(V\hat{y}(t)).
\end{equation}
The system of \eqref{Chapter2:problem} is now completely reduced in dimension. The linear part is reduced to $k \ll n$ and the nonlinear part to $m \ll n$. 

How the coefficients $c(y(t))$ will be determined, will be explained in Chapter \ref{Chapter2:deim}, along with an explanation of the DEIM-algorithm and an short error analysis.
How the two basis are constructed using the POD method will be explained in then next chapter.


\section{Proper Orthogonal Decomposition (POD)}\label{Chapter2:pod}
Consider a set of snapshots $S= \{y_1,\dots,y_{n_s}\} \subset \mathbb{R}^n$ and the corresponding matrix $Y= 
[y_1,\dots,y_{n_s}] \in \mathbb{R}^{n \times n_s}$. The POD method constructs an orthonormal basis in the space spanned by $S$.
Let $r = rank\{Y\}$, then there exist $\{v_i\}^k_i \subset \mathbb{R}^n$ orthonormal basis vectors, for a $k \leq r$.

\begin{mydef} \textbf{(SVD):}
Let $\mathbb{A} \in \mathbb{R}^{m \times n}$ a matrix, then there exist two orthogonal matrices  
\begin{equation*}
 U=[\varphi_1 | \dots | \varphi_n] \in \mathbb{R}^{m \times m}, \qquad Z=[\varPsi_1|\dots|\varPsi_m] \in \mathbb{R}^{n \times n}
\end{equation*}
such that
\begin{equation}
 \mathbb{A} = U\Sigma Z^T, \; mit \; \Sigma = diag(\sigma_1,\dots,\sigma_r) \in \mathbb{R}^{m \times n}
\end{equation}
and $\sigma_1 \geq \dots \geq \sigma_r > 0$, for r = min(m,n), $\mathbb{A}\varPsi_i = \sigma_i \varphi_i$ and $\mathbb{A}^T \varphi_i = \sigma_i \varPsi_i$ with $i=1,\dots,r$.
\end{mydef}

The SVD of $Y = U\Sigma Z^T $
with orthogonal matrices $U=[\varphi_1 | \dots | \varphi_n] \in \mathbb{R}^{n \times n}$ and
$Z=[\varPsi_1|\dots|\varPsi_m] \in \mathbb{R}^{m \times m}$ and the diagonal matrix 
$\Sigma = diag(\sigma_1, \dots ,\sigma_r)$ with singular values $\sigma_1 \geq \dots \geq \sigma_r$
provides these  basis vectors. Note that the diagonal matrix $\Sigma$ is not square, thus, $diag(\sigma_1,\dots,\sigma_r)$ provides 
a rectangular matrix with entries $a_{ij} = \sigma_i$ for $i=j$ and $a_{ij} = 0$ for $i \neq j$. 

For a $k \leq m$ the POD-basis is defined as the set of the first $k$ left singular vectors $\varphi_1,\dots,\varphi_k$
of $U$. Let $V = [\varphi_1,\dots,\varphi_k] \in \mathbb{R}^{n \times k}$ be the matrix of these POD-basis vectors.
An approximation of a snapshot $y_j$ in the span of $Y$ is therefore $Vc$
with coefficients $c \in \mathbb{R}^k$. The Galerkin orthogonality of the residual $y_j - Vc$ 
to the span of $V$ gives $V^T(y_j - Vc) = 0$, hence $y_j \approx VV^Ty_j$. 
The POD-basis provides a optimal orthogonal basis with regard to 
the sum of the quadratic error of the approximation of the snapshots.

\newpage

\subsection{Error Bound for POD}
\begin{mytheo} \textbf{(POD-error bound, \cite[p.~125]{ROM_book2}):}

Let $\mathbb{V}_k = \{ W \in \mathbb{R}^{n \times k} : W^TW = I_k\} $ be the set of all k-dimensional orthonormal bases 
and $V \in \mathbb{R}^{n \times k}$  the matrix of basis vectors provided by the POD-method related to 
the snapshots $Y = \{y_1,\dots,y_m\}$, then the following holds
\begin{equation}\label{ch2:error_bound_pod}
 \sum_{j=1}^m\parallel y_j - VV^Ty_j\parallel_2^2 =\min_{W \in \mathbb{V}_k} 
 \sum_{j=1}^m\parallel y_j - WW^Ty_j\parallel_2^2 = \sum_{i=k+1}^r \sigma^2_i.
\end{equation}
\end{mytheo}

\begin{mytheo}
  \textbf{(Schmidt-Eckart-Young, \cite[p.~118]{ROM_book2}):}
Let $A \in \mathbb{R}^{m \times n}$ be a matrix of rank $p$, then there exist two orthogonal matrices 
$U=[\varphi_1 | \dots | \varphi_n] \in \mathbb{R}^{n \times n}$ and
$Z=[\varPsi_1|\dots|\varPsi_m] \in \mathbb{R}^{m \times m}$, so that $A = U\Sigma Z^T$ with the diagonal-matrix 
$\Sigma = diag(\sigma_1,\dots,\sigma_r)$ and
the singular values $\sigma_1 \geq \dots \geq \sigma_r \geq 0$ for r=min(m,n).
The matrix
\begin{equation}
 A_k = \sum_{i=1}^k {\sigma_i \varphi_i \varPsi_i^T}, \qquad 0 \leq k \leq p
\end{equation}
satisfies the property
\begin{equation}\label{sey}
 \| A-A_k\|_F = \min_{\substack{B\in \mathbb{R}^{m \times n} \\ rank(B) \leq k}} \| A-B\|_F = \sqrt{\sum_{i=k+1}^p \sigma_i^2}.
\end{equation}
\end{mytheo}

\textbf{Proof of Theorem 1, \cite[p.~125]{ROM_book2}:}
By Theorem 2, the best rank $k$ approximation of $Y$ is given by
\begin{equation*}
 Y_k = \sum_{i=1}^k {\sigma_i \varphi_i \varPsi_i^T}.
 \end{equation*}
With $\varPsi_i = \frac{1}{\sigma_i}Y^T\varphi_i$, which results from the definition of the SVD, follows
\begin{equation*}
 Y_k = \sum_{i=1}^k {\sigma_i \varphi_i (\frac{1}{\sigma_i}Y^T\varphi_i)^T} 
 = \sum_{i=1}^k { \varphi_i \varphi_i^T Y} =VV^TY.
\end{equation*}
Moreover, for a $W \in \mathbb{V}_k$
\begin{equation*}
 \sum_{j=1}^m\parallel y_j - WW^Ty_j\parallel_2^2 
 = \parallel Y- WW^TY \parallel_F^2,
\end{equation*}
because of the definition of the Frobenius norm.
And with (\ref{sey}) it is
\begin{equation*}
 \| Y-Y_k\|_F^2 = \min_{\substack{B\in \mathbb{R}^{m \times n} \\ rank(B) \leq k}} \| Y-B\|_F^2 
 \leq \min_{W \in \mathbb{V}_k} \parallel Y - WW^TY\parallel_F^2.
\end{equation*}
As $Y_k = VV^TY$ follows
\begin{equation*}
\parallel Y - VV^TY\parallel_F^2 =
\min_{W \in \mathbb{V}_k} \parallel Y - WW^TY\parallel_F^2 = \sum_{i=k+1}^r \sigma^2_i.
\end{equation*}

For the quality of the approximation of the POD reduced model
the choice of the dimension $k$ of th POD basis is importent. There exist no theoretical bound
for the approximation error depending on $k$.
According to \cite{ROM_book2} in practice the ratio of the first $k$ singular values to all singular values, i.e
\begin{equation}
 I(k) =\frac{ \sum_{i=1}^k \sigma^2_i}{\sum_{i=1}^r \sigma^2_i}
\end{equation}
is used as an error condition for selcting a dimension $k$  heuristically.
For a given approximation error $\epsilon_{POD}$ 
\begin{equation}\label{Chapter2:eq:error_funcion_pod}
  I(k) = \frac{ \sum_{i=1}^k \sigma^2_i}{\sum_{i=1}^r \sigma^2_i} \leq 1- \epsilon_{POD}^2
\end{equation}
can be used to determine $k$. 

\section{Discrete Empirical Interpolation Method (DEIM)}
\label{Chapter2:deim}
In order to reduce the nonlinear part in \eqref{Chapter2:problem_pod_dim}, a low-dimensional approximation
of a nonlinear function $q(t): \mathbb{D} \longmapsto \mathbb{R}^n$ is needed, i.e.
\begin{equation}\label{Chapter2:deim_approx}
 q(t) \approx Uc(t),
\end{equation}
where $c(t) \in \mathbb{R}^m$ is the corresponding coefficient vector and $U \in \mathbb{R}^{n \times m}$ is
a matrix obtain from the application of the POD approach only on the nonlinear terms. Indeed, it is received from an SVD of a snapshots
matrix of the nonlinearity.
To determine the vector $c(t)$ $m$ distinguished rows are selected from the over-determined system 
$q(t) = Uc(t)$. The selection of the rows can be done with a matrix 
\begin{equation}
 P = [e_{p1},\dots,e_{pm}] \in \mathbb{R}^{n \times m},
\end{equation}
where $e_{pi} = [0,\dots0,1,0,\dots,0]^T \in \mathbb{R}^n$ is the $pi$-th unit vector, for $i= 1,\dots,m$.
By multiplying $P^T$ on $q(t) = Uc(t)$ $m$ rows are selected. If $P^TU$ is non-singular,
the vector $c(t)$ can be determined from
\begin{equation}
 c(t) = (P^TU)^{-1}P^Tq(t)
\end{equation}
and the approximation \eqref{Chapter2:deim_approx} becomes
\begin{equation}
  q(t) \approx Uc(t) = U(P^TU)^{-1}P^Tq(t)
\end{equation}
The DEIM thus requires to construct a projection basis $U$ and interpolation indices $\{ p_1,\dots,p_m\}$ used for $P$.
The projection basis $U$ is constructed by using the POD on a nonlinear set of snapshots 
$Y_{q} = [q(t_1),\dots,q(t_m)]$ obtained from the 
high order system. These snapshots are the evaluations of the nonlinear function $q(t)$ on $t_1,\dots,t_{n_s}$.
They can be obtained together with the snapshots for the full-system POD, because they are already computed there.
Thus only the SVD has to be computed to obtain $U$.

The interpolation indices are iteratively selected by the greedy algorithm from the 
basis $U$ (cf. Algorithm \ref{Chapter2:deim_algorithm}). The algorithm selects iteratively $m$ indices $\{ p_1,\dots,p_m\}$, 
where $p_i \in \{1,\dots,n\}$ for $i = 1,\dots,m$. This minimizes the interpolation error over the
snapshot set measured in the maximum norm. It starts by selecting the index of the biggest 
entry of the first column of $U$ and thus the first basis vector, which corresponds to the 
most dominant singular value of the SVD. Further indices are selected in the way that they correlate to the 
index of the larges entry of the residual $r= u_i - Uc$. The residual $r$ is the error between the basis
vector $u_i$ and its projected approximation $Uc$. Since the columns of $U$ are linear independent, $r$ is
a nonzero vector in each step. Thus, an entry with the larges magnitude can always be selected.

\begin{algorithm}
\caption{DEIM , \cite{Chaturantabut2010Deim}}\label{Chapter2:deim_algorithm}
\begin{algorithmic}[1]
   \item \textbf{Input:}{$\{u_i\}_{i=1}^m \subset \mathbb{R}^n$ }
   \item \textbf{Output:}{$\vec{\varrho} = \{\varrho_1,\dots,\varrho_m\}^T \in \mathbb{N}^m$}
   \State $ \varrho_1 = argmax\{|u_1|\}$
   \State $U = [u_1], P = [e_{\varrho_1}], \vec{\varrho} = [\varrho_1]$
   \For{$i=2 $ to $m$}
      \State \textbf{Solve}   $(P^TU)c= P^Tu_i$ 
      \State $r = u_i - Uc$
      \State $\varrho_i = argmax\{|r|\}$
      \State $U \gets [U | u_i], P \gets [P | e_{\varrho_i}], \vec{\varrho} \gets \left[\begin{array}{r}\varrho\\ \varrho_i \end{array}\right] $
   \EndFor
\end{algorithmic}
\end{algorithm}

\newpage
In line 6 of Algorithm \ref{Chapter2:deim_algorithm} the linear system 
$(P^TU)c= P^Tu_i$ must be solved. This is only possible if $P^TU$ is always non-singular, as is shown in \cite{Chaturantabut2010Deim}.
The whole procedure has a complexity  of $O(m^4+mn)$ (cf. \cite[Chapter 2.2.6]{Chaturantabut2010Deim}), where the dimension $m$ of the basis $U$ is
small.

Formally the DEIM approximation is defined in \cite{Chaturantabut2010Deim} as follows.
\begin{mydef}
 Let $f:D \longmapsto \mathbb{R}^n$ be a nonlinear vector valued function with $D \subset \mathbb{R}^d$ for a $d \in \mathbb{N}$.
 Let $\{u_l\}_{l=1}^m \subset \mathbb{R}^n$ be a set of linear independent vectors.
 For an $x \in D$ the DEIM approximation of order $m$ in the space spanned by $\{u_l\}_{l=1}^m$ is given by:
 \begin{equation}\label{deim_app}
  \hat{f}(x) := U(P^TU)^{-1}P^Tf(x)
 \end{equation}
 where $ U = [u_1 | \dots | u_m] \in \mathbb{R}^{n \times m}$ is the basis, provided by the POD method. Moreover $P = [e_{p_1} | \dots | e_{p_m}] \in \mathbb{R}^{n \times m}$,
 where $\{p_1,\dots,p_m\}$ is the output of the DEIM algorithm with $\{u_l\}_{l=1}^m$ as input vectors.
\end{mydef}
It can be shown that $\hat{f}(x)$ is a interpolation of the original $f(x)$. It holds
\begin{equation*}
 P^T\hat{f}(x)= P^T(U(P^TU)^{-1}P^Tf(x)) = (P^TU) (P^TU)^{-1}P^Tf(x)=P^Tf(x).
\end{equation*}
The functions $\hat{f}(x)$ and $f(x)$ are exactly the same at the DEIM points $\{p_1,\dots,p_m\}$.


\subsection{Error Bound for DEIM}
\begin{mytheo}\textbf{(DEIM-error),\cite{Chaturantabut2010Deim}:}
Let $f \in \mathbb{R}^n$ be a vector, let $\{u_l\}_{l=1}^m \subset \mathbb{R}^n$ be a set of linear 
independent orthogonal vectors.
The DEIM approximation of order $m \leq n$ of $f$ in the space spanned by $\{u_l\}_{l=1}^m$ is 
$\hat{f} = \mathbb{P}f$, where $\mathbb{P} = U(P^TU)^{-1}P^T$, $ U = [u_1 | \dots | u_m] \in \mathbb{R}^{n \times m}$ and $P = [e_{p_1} | \dots | e_{p_m}] \in \mathbb{R}^{n \times m}$,
 with $\{p_1,\dots,p_m\}$ being the output of the DEIM-algorithm with input $\{u_l\}_{l=1}^m$. Then the following holds
 \begin{equation}
  \parallel f-\hat{f} \parallel_2 \leq C_m \varepsilon(f),
 \end{equation}
where $C_m = \parallel (P^TU)^{-1}\parallel_2$ and $\varepsilon(f) = \parallel f - UU^Tf \parallel_2$. 
\end{mytheo}

\textbf{Proof of Theorem 3, \cite{Chaturantabut2010Deim}:}
Let $\hat{f}(x)$ be the DEIM approximation. To find an error bound in the 2-norm for
 $\parallel f-\hat{f} \parallel_2$ look at the best approximation of $f$ in the space spanned by $U$,
 in particular $f_* = UU^Tf$. Then it holds
\begin{equation}
 f = (f- f_*) +f_* = w +f_* \qquad \text{with} \qquad w= f-f_*=f-UU^Tf
\end{equation}
and also
\begin{equation}
 \hat{f} = \mathbb{P}f = \mathbb{P}(w +f_*) =  \mathbb{P}w + f_*,
\end{equation}
because $\mathbb{P} f_* = f_*$.
Therefore
\begin{equation}\label{Chapter2:eq:deim_error}
 \parallel f- \hat{f} \parallel_2 = \parallel w +f_* - (\mathbb{P}w + f_*) \parallel_2 \leq \parallel I - \mathbb{P}\parallel_2 \parallel w \parallel_2
\end{equation}

with $\parallel I - \mathbb{P}\parallel_2 = \parallel  \mathbb{P}\parallel_2$ as shown in
\cite{Szyld2006Norm} for any projector-matrix $P \neq 0$. From \ref{Chapter2:eq:deim_error} follows

\begin{equation}
 \parallel f- \hat{f} \parallel_2 \leq \parallel I - \mathbb{P}\parallel_2 \parallel w \parallel_2 = \parallel  \mathbb{P}\parallel_2 \parallel w \parallel_2
\end{equation}
and with $ \parallel \mathbb{P} \parallel_2= \parallel U(P^TU)^{-1}P^T  \parallel_2 = \parallel (P^TU)^{-1}  \parallel_2$.
Because $U$ and $\mathbb{P}$ are both orthogonal it holds
\begin{equation}
 \parallel f- \hat{f} \parallel_2 \leq \parallel (P^TU)^{-1}  \parallel_2 \parallel w \parallel_2
\end{equation}

Overall it holds 
\begin{equation}
 \parallel f- \hat{f} \parallel_2 \leq \parallel \mathbb{P} \parallel_2 \parallel w \parallel_2  =\parallel (P^TU)^{-1} \parallel_2  \parallel f-UU^Tf \parallel_2 =C_m \varepsilon(f).
\end{equation}

The factor $(P^TU)^{-1}$ depends on the matrix $P$, which is constructed out of the DEIM indices 
$\{p_1,\dots,p_m\}$. That is the reason the DEIM algorithm aims to select an index that limits the 
growth of $\parallel (P^TU)^{-1} \parallel_2$ and thus the error of $\parallel f- \hat{f} \parallel_2$.
Chaturantabut shows in \cite{Chaturantabut2010Deim} a recursive formula for $(P^TU)^{-1}$. This proofs the minimization 
of this error and also the non-singularity of $(P^TU)^{-1}$, which is needed in the Algorithm \ref{Chapter2:deim_algorithm}.
Moreover, it can be shown that $C_m$ can be bound with 
\begin{equation}
 C_m \leq \frac{(1 + \sqrt{2n})^{m-1}}{\parallel u_1 \parallel_\infty}.
\end{equation} 
This bound is very pessimistic and it will grow more rapidly than the actual matrix $\parallel (P^TU)^{-1} \parallel_2$. In \cite{Chaturantabut2010Deim} it is advised to use the matrix norm as a posteriori error estimation, than the priori estimation
$C_m$, because the matrix $(P^TU)^{-1}$ is usually small. 

The term $\varepsilon(f) = \parallel f - UU^Tf \parallel_2$ depends on $f$. It changes for ever new $f$, hence 
it is expensive to compute. It is desirable to have an easily computable estimation.
In \cite{Chaturantabut2010Deim}, the approximation $\varepsilon(f)\approx \sigma_{m+1}$ is used, thus the singular value corresponding to the first left singular vector that has not been taken into the DEIM base matrix $U$.
This estimate is purely heuristic and is not proved in general, but for the numerical examples in \cite{Chaturantabut2010Deim} it does seem to provide a reasonable qualitative estimate of the
expected error. The DEIM error should be bound like this, if the trajectories are attracted to a low-dimensional subspace, this means they lie nearly in the space spanned by the snapshots.
In this work the error bound approximation 
\begin{equation}\label{ch2:error_bound_deim}
  \bar{\mathcal{E}_*} = \parallel f-\hat{f} \parallel_2 \leq C_m \varepsilon(f) \lessapprox \parallel (P^TU)^{-1} \parallel_2 \sigma_{m+1}
\end{equation}
 will be used for the numerical experiments. 



\section{Overview of POD-DEIM approach}
\label{overview}

To recall what the POD-DEIM approach requires to reduce a system like \eqref{problem:ODE} and how the workflow will be, look at Figure \ref{Chapter2:workflowdeim}. To see an overview over the complexity of the different parts look at Table \ref{Chapter2:tablecomplex}.
The generation of linear and nonlinear snapshots can usually be done in one model evaluation. The computation of the SVD for a large matrix is quite costly. Thus if the number of snapshots $n_s$ grows, it is more efficient to compute
a thin SVD and the singular vectors and the singular values iteratively and parallel with for example with ARPACK \cite{arpack} or SLEPc \cite{slepc-toms}. 
The computational work in Table \ref{Chapter2:tablecomplex} has only to be done once, to get the reduced system. The matrices $\hat{A}$ and  $\mathbb{P}$ are precomputed and stored and reused 
when solving the reduced system. 
\begin{figure}[H]
\begin{center}





\begin{tikzpicture}[node distance = 3cm, auto]
    % Place nodes
    \node [block] (init) {Generation of full system $S=\{ y_1,\dots,y_{ns}\}$ and nonlinear $S_q =\{ q_1,\dots,q_{ns}\}$ snapshots};
    \node [block, right of=init ,node distance=5cm] (svd) {Compute SVD for $S$ and $S_q$};
    \node [block, right of=svd,node distance=5cm] (identify) {Construction of Bases $V \in \mathbb{R}^{n \times k}$ and $U \in \mathbb{R}^{n \times m}$ with POD};
    \node [block, below of=identify, node distance=4cm] (evaluate) {Computation of $m$ DEIM indices with Algorithm \ref{Chapter2:deim_algorithm}};
    \node [block, left of=evaluate, node distance=5cm] (update) {Precompute: \\$\hat{A} = V^TAV$\\  $\mathbb{P}=V^TU(P^TU)^{-1}$};
    \node [blocklong, below of=svd, node distance=8cm] (stop) {Solve the reduced system: \\$\hat{y}(0) = V^T y_0 $\\ $
  \frac{d}{dt}\hat{y}(t) = \underbrace{\hat{A}}_{k \times k}\hat{y}(t)+ \underbrace{\mathbb{P}}_{k \times m} q(P^TV\hat{y}(t))$};
     \node [block, below of=init, node distance=12cm] (recovering) {Recovering for visualization and post processing: \\ $y = V\hat{y}$ };
     \node [block, below of=identify, node distance=12cm] (error) {Estimating the approximation error:\\ $\parallel y - V\hat{y}\parallel_2$  };
    % the labels on the left
    \begin{scope}[xshift=-3cm,yshift=-2cm]
        \node[mylabel] (offline)  {\rotatebox{90}{Offline}};
        \node[mylabel, below of=offline, node distance=8cm] (online) {\rotatebox{90}{Online}};
    \end{scope}
    % Draw edges
    \path [line] (init) -- (svd);
    \path [line] (svd) -- (identify);
    \path [line] (identify) -- (evaluate);
    \path [line,dashed] (update) -- (stop) ;
    \path [line] (evaluate) -- (update);
    \path [line] (stop) -- (recovering);
    \path [line] (stop) -- (error);

\end{tikzpicture}
\end{center}
\caption{An overview on the workflow of model reduction.}
 \label{Chapter2:workflowdeim}
\end{figure}


\begin{table}[H]
\begin{center}


$\begin{array}{|c|c|}
\hline
\text{compute Snapshots} & \text{Problem dependent} \\
  \text{SVD for POD-basis} & \mathcal{O}(nn_s^2) \\
  \text{DEIM algorithm for m interpolation indices} & \mathcal{O}(m^4+nm) \\
  \text{Precompute: } \hat{A} = V^TAV & \mathcal{O}(NNZk+nk^2)  \text{ for a sparse A} \\
  \text{Precompute: } \mathbb{P} = V^TU(P^TU)^{-1} & \mathcal{O}(nkm + m^2n + km^2 + m^3)\\
\hline
\end{array}$
\end{center}
\caption{Computational Complexity of POD-DEIM}
 \label{Chapter2:tablecomplex}
\end{table}


